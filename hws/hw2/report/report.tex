\documentclass[11pt,a4paper]{article}
\usepackage[utf8]{inputenc}
\usepackage{amsmath}
\usepackage{amsfonts}
\usepackage{amssymb}
\usepackage{graphicx}
\usepackage[left=2cm,right=2cm,top=2cm,bottom=2cm]{geometry}
\begin{document}


\begin{titlepage}

\newcommand{\HRule}{\rule{\linewidth}{0.5mm}} % Defines a new command for the horizontal lines, change thickness here

\center % Center everything on the page
 
%----------------------------------------------------------------------------------------
%	HEADING SECTIONS
%----------------------------------------------------------------------------------------

\textsc{\LARGE Université Catholique de Louvain}\\[1.5cm] % Name of your university/college
\textsc{\Large Cloud Computing}\\[0.5cm] % Major heading such as course name
\textsc{\large LINGI2261}\\[0.5cm] % Minor heading such as course title

%----------------------------------------------------------------------------------------
%	TITLE SECTION
%----------------------------------------------------------------------------------------

\HRule \\[0.4cm]
{ \huge \bfseries Assignment 2 : Scaling Ribbit}\\[0.4cm] % Title of your document
\HRule \\[1.5cm]
 
%----------------------------------------------------------------------------------------
%	AUTHOR SECTION
%----------------------------------------------------------------------------------------



% If you don't want a supervisor, uncomment the two lines below and remove the section above
\Large \emph{Author:}\\
Eddy \textsc{Ndizera}\\
Ivan \textsc{Ahad} \\[3cm] % Your name

%----------------------------------------------------------------------------------------
%	DATE SECTION
%----------------------------------------------------------------------------------------

{\large \today}\\[3cm] % Date, change the \today to a set date if you want to be precise

%----------------------------------------------------------------------------------------
%	LOGO SECTION
%----------------------------------------------------------------------------------------

%\includegraphics{Logo}\\[1cm] % Include a department/university logo - this will require the graphicx package
 
%----------------------------------------------------------------------------------------

\vfill % Fill the rest of the page with whitespace

\end{titlepage}

\section{Explain the high level design of your solutions to the three scalability challenges of the application}

\section{A description of your database design and data model. In particular, how did you decide to store user relations and tweets}
Here are the different database tables that we used for the project : 
\begin{itemize}
\item A table of users representing a user in general, with its full name and its username as primary key
\item A table containing all the tweets, with the ID of each of them, the author who wrote every tweet, a timestamp "created\_at" that tells when the tweet has been posted, and the message that the tweet contains. 
\item A table containing the timeline, that is all the tweets coming from all the users that an account follows. In opposition to the table of tweets, the primary key here is the username, corresponding to the current account for which all the tweets of all its users are in the timeline. It also contains the author of each tweet, the ID of the tweets, the timestamp, and the text.
\item A table called "Users\_src\_dest" that contains a source, and a destination. In this table, for each object in the table, a "src associated to a "dest" means that the destination is an account that follows the source, with the source-dest pair as primary key, since only the pair needs to be unique, the source might be duplicated but with a different destination. This table is very important to be able to retrieve all the followers of a particular account, just by using the username as a source.
\item Similarly, we created a table called "Users\_dest\_src" where a destination account follows a source account. It is useful to retrieve all the accounts that a particular is currently following. The primary key is a pair dest-source, for the same reason as the previous table. Those two last tables are the ones we used in particular to store the user relations.
\end{itemize}

\section{Describe the challenges you encountered, if any}
As we struggled finding two other members to be in a team of 4, every aspect of the project was more challenging.
\section{Describe how to run your application}


\end{document}